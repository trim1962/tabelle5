% !TeX spellcheck = it_IT
% !TeX encoding = UTF-8
% !TeX root = tabellequinto.tex
\chapter{Limiti}
\section{Limite finito per x che tende a valore finito}
Supponiamo di avere una funzione $\funzione{f}{D}{\R}$ definita \[f(x)=3x+1\] Dove $D$ è il suo dominio. Costruiamo la seguente tabella della funzione per valori di $x$ prossimi a uno.
\begin{center}
\begin{tabular}{@{}L*{7}{c}}
	\toprule
&&	\multicolumn{2}{R}{\longrightarrow}&&\multicolumn{2}{L}{\longleftarrow}\\
x	& 0.9 & 0.99 & 0.999&1 &1.001  &1.01  & 1.1 \\
\cmidrule(r){2-8} 
f(x)	&3.7  & 3.97 &3.997 & & 4.003 & 4.03 & 4.3 \\ 
\cmidrule(r){2-8} 
\abs{f(x)-4}& 0.3 &0.03  &0.003  &  & 0.003 &0.03& 0.3 \\ 
\bottomrule
\end{tabular}\captionof{table}{Limite finito per x che tende a valori finito}
\end{center}
Guardando la tabella, si nota che per valori di $x$ prossimi a uno la funzione assume valori vicini al  quattro. Dalla tabella risulta che la differenza tra il valore della funzione e il quattro può essere resa piccola a piacere.\par Quanto più piccola è questa differenza fissata tra la funzione e il numero, esiste un intorno dell'uno per cui, preso un valore appartenente a questo intorno, la differenza fra la funzione e il 4,  è minore di quanto fissato. 

Se succede questo si dice che \[\lim_{x\to 1}3x+1=4 \]

\begin{equation*}
\forall\; I(4)\; \exists\; I(1) : \forall x\in I(1)-\lbrace 1\rbrace \longrightarrow f(x)\in I(4)
\end{equation*}
In generale \begin{equation*}
\lim_{x\to x_0}f(x)=l
\end{equation*}
\begin{equation*}
\forall\; I(l)\; \exists\; I(x_0) : \forall x\in I(x_0)-\lbrace x_0\rbrace \longrightarrow f(x)\in I(l)
\end{equation*}
 \section{Limite sinistro e limite destro}
 Per spiegare il concetto usiamo la seguente funzione funzione vale \[ f(x)=\begin{cases}
 +1&x>0\\
 -1&x<0\\
 0&x=0
 \end{cases}\]
 Il grafico della funzione è il seguente
  \begin{center}
 	\includestandalone{grafici/FunzioneSegno}
 	\captionof{figure}{Funzione segno}
 \end{center}\index{Funzione!segno}
il limite per  x che tende a zero di questa funzione non esiste. Ma esistono \[\lim_{x\to 0^+}f(x)=1 \]
e \[\lim_{x\to 0^-}f(x)=-1 \]
\section{Limite infinito per x che tende a valore finito}
Supponiamo di avere una funzione $\funzione{f}{D}{\R}$ definita \[f(x)=\dfrac{1}{x^2}\]Dove $D$ è il suo dominio. Costruiamo la seguente tabella della funzione per valori di $x$ prossimi a zero.
\begin{center}
	\begin{tabular}{@{}L*{13}{c}}
		\toprule
		&&&&&	\multicolumn{2}{R}{\longrightarrow}&&\multicolumn{2}{L}{\longleftarrow}\\
		x& -1 & -0.9 & -0.5&-0.2&-0.1&-0.01&0&0.01&0.1&0.2&0.5&0.9&1 \\
		\cmidrule(r){2-14} 
	f(x)	&1&1.12&4&25&100&10000&&10000&100&25&4&0.12&1\\
		\bottomrule
	\end{tabular}\captionof{table}{Limite infinito per x che tende a valore finito}
\end{center}
Per valori prossimi allo zero la funzione assume valori sempre maggiori. Possiamo dire che valori prossimi allo zero, comunque fissiamo un valore $M>0$ esiste un intorno dello zero $I(0)$ tale comunque preso $x$ appartenente a questo intorno, escluso lo zero, $f(x)>M$

Se succede questo si dice che \[\lim_{x\to 0}\dfrac{1}{x^2}=\infty \]
\begin{equation*}
\forall\; I(\infty)=(M,\infty)\; \exists\; I(0) : \forall x\in I(0)-\lbrace 0\rbrace \longrightarrow f(x)\in I(\infty)
\end{equation*}
In generale \begin{equation*}
\lim_{x\to x_0}f(x)=\infty
\end{equation*}
\begin{equation*}
\forall\; I(\infty)\; \exists\; I(x_0) : \forall x\in I(x_0)-\lbrace x_0\rbrace \longrightarrow f(x)\in I(\infty)
\end{equation*}
\section{Limite finito per x che tende a valore infinito}
Supponiamo di avere una funzione $\funzione{f}{D}{\R}$ definita \[f(x)=\dfrac{3x-2}{x+4}\] Dove $D$ è il suo dominio.
Costruiamo la seguente tabella per valori di x crescenti.
\begin{center}
	\begin{tabular}{@{}L*{6}{c}}
		\toprule&
			\multicolumn{2}{L}{\longrightarrow}&&\multicolumn{1}{L}{\longrightarrow}\\
		x& 1 & 10 &100&1000&10000&100000\\
		\cmidrule(r){2-7} 
	f(x)	&0.2&2&2.7&2.98&2.99&2.9998\\
	\cmidrule(r){2-7} 
	\abs{3-f(x)}&2.8&1&0.3&0.02&0.01&0.0002\\
		\bottomrule
	\end{tabular}\captionof{table}{Limite finito per x che tende a  infinito}
\end{center}
All'aumentare dei valori di $x$ la funzione sempre più prossima a quattro. Inoltre per quanto possa essere piccola la differenza tra la funzione e il valore limite è possibile trovare un valore di x per cui questa differenza è ancora minore.
Per ogni intorno di tre esiste un intorno dell'infinto tale che per ogni valore di questo intorno $f(x)$ appartiene all'intorno del tre.
Se succede questo si dice che \[\lim_{x\to \infty}\dfrac{3x-2}{x+4}=3 \] 
\begin{equation*}
\forall\; I(3)\; \exists\; I(\infty) : \forall x\in I(\infty) \longrightarrow f(x)\in I(3)
\end{equation*}
In generale \begin{equation*}
\lim_{x\to \infty}f(x)=l
\end{equation*}
\begin{equation*}
\forall\; I(l)\; \exists\; I(\infty) : \forall x\in I(\infty) \longrightarrow f(x)\in I(l)
\end{equation*}
\section{Limite infinito per x che tende a valore infinito}
Supponiamo di avere una funzione $\funzione{f}{D}{\R}$ definita \[f(x)=x^2\] Dove $D$ è il suo dominio.
Costruiamo la seguente tabella per valori di x crescenti.
\begin{center}
	\begin{tabular}{@{}L*{5}{c}}
		\toprule
		\multicolumn{1}{R}{\longrightarrow}&&&\multicolumn{1}{R}{\longrightarrow}\\
		x& 0 & 1 &2&10&100\\
		\cmidrule(r){2-6} 
		f(x)	&0&1&4&100&10000\\
		\bottomrule
	\end{tabular}\captionof{table}{Limite infinito per x che tende a  infinito}
\end{center}
All'aumentare di $x$ la funzione assume valori sempre più grandi
Quindi per ogni intorno del limite esiste un intorno della x per cui $f(x)$ appartiene all'intorno del limite. Possiamo dire che  Se succede questo si dice che \[\lim_{x\to \infty}x^2=\infty \]
In generale \begin{equation*}
\lim_{x\to \infty}f(x)=\infty
\end{equation*}
\begin{equation*}
\forall\; I(\infù)\; \exists\; I(\infty) : \forall x\in I(\infty) \longrightarrow f(x)\in I(\infty)
\end{equation*}
  