\chapter{Limiti}
\section{Limite finito per x che tende a valore finito}
Supponiamo di avere una funzione $\funzione{f}{\R}{\R}$ definita \[f(x)=3x+1\] Costruiamo la seguente tabella della funzione per valori di $x$ prossimi ad uno.
\begin{center}
	
\begin{tabular}{@{}Lccccccc}
	\toprule
&&	\multicolumn{2}{R}{\longrightarrow}&&\multicolumn{2}{L}{\longleftarrow}\\
x	& 0.9 & 0.99 & 0.999&1 &1.001  &1.01  & 1.1 \\
\cmidrule(r){2-8} 
f(x)	&3.7  & 3.97 &3.997 & & 4.003 & 4.03 & 4.3 \\ 
\cmidrule(r){2-8} 
\abs{f(x)-4}& 0.3 &0.03  &0.003  &  & 0.003 &0.03& 0.3 \\ 
\bottomrule
\end{tabular}\captionof{table}{Limite finito per x che tende a valori finito}
\end{center}
Guardando la tabella, si nota che per valori di $x$ prossimi a uno la funzione assume valori prossimi al quattro. Dalla tabella risulta che la differenza tra il valore della funzione e il quattro può essere resa piccola a piacere. Quanto più piccola si vuole la differenza tra la funzione e il numero esiste un intorno del punto dell'uno per cui preso un valore appartenente a questo intorno, la differenza è minore di quanto fissato. Se succede questo si dice che \[\lim_{x\to 1}3x+1=4 \]
In generale \[ \forall \epsilon>0\; \exists\; \delta_\epsilon>0 \; |\; \forall x \; 0<\abs{x-x_0}<\delta \Rightarrow \abs{f(x)-l}<\epsilon \]
 \[\lim_{x\to x_0}f(x)=l \]
 \section{Limite sinistro e limite destro}
 Per spiegare il concetto usiamo la seguente funzione\index{Funzione!segno} segno\[f(x)=\dfrac{\abs{x}}{x}\]Questa funzione vale \[ f(x)=\begin{cases}
 +1&x>0\\
 -1&x<0\\
 \text{Indefinita}&x=0
 \end{cases}\]