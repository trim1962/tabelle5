
% see http://www.tex.ac.uk/cgi-bin/texfaq2html?label=noroom
\documentclass[a4paper,oneside]{book}
\NeedsTeXFormat{LaTeX2e}
\ProvidesPackage{base}[2019/07/09]
\RequirePackage{amsmath}
\RequirePackage{amsthm}
% 10/09/2017 :: 10:07:45 :: \RequirePackage{amsfonts}
\RequirePackage{amssymb}
\RequirePackage{ifluatex}
\ifluatex
% Lua(La)TeX
\RequirePackage[italian]{babel}
%\RequirePackage{polyglossia}   
%\setmainlanguage[spelling=new,babelshorthands=true]{german}
\RequirePackage{fontspec}
\defaultfontfeatures{Ligatures=TeX}
\RequirePackage{unicode-math} 
\unimathsetup{math-style=TeX}
\setmainfont[Ligatures=TeX]{Constantia} 
\setsansfont{Calibri}
\setmonofont{Consolas}
\setmathrm{Cambria Math}
\setmathfont{Cambria Math}
\else
%default: pdf(La)TeX
\RequirePackage[italian]{babel}
%\RequirePackage{concmath}

%\RequirePackage{newtxtext,newtxmath}


 % load vector font
\RequirePackage[T1]{fontenc} % font encoding
\RequirePackage[utf8]{inputenc} % input encoding
%\RequirePackage{noto}
\RequirePackage[largesc]{newtxtext} %
\RequirePackage[varqu,varl]{zi4}% inconsolata
\RequirePackage{cabin}% sans serif
\RequirePackage[vvarbb]{newtxmath}
\useosf % use oldstyle figures except in math

\fi
\RequirePackage[babel=true]{microtype}
\RequirePackage{geometry}
%\RequirePackage{textcomp}
% 10/09/2017 :: 10:16:38 :: \geometry{top=2cm,bottom=2cm,left=2cm,right=2cm}

\geometry{top=1.5cm,bottom=1.5cm}
\NeedsTeXFormat{LaTeX2e}
\ProvidesPackage{grafica}[2019/07/09]
\RequirePackage{forest}
\RequirePackage{tkz-euclide}
\usetkzobj{all}
\RequirePackage{circuitikz}
\RequirePackage{tkz-tab}
\RequirePackage{tkz-linknodes}
\usetikzlibrary{calc}
%\usetikzlibrary{hobby}
\RequirePackage{pgfplots}
 \pgfplotsset{compat=newest}
\usetikzlibrary{decorations}
\usetikzlibrary{shapes,arrows,trees,positioning,through,intersections}
\usetikzlibrary{shapes.misc,shapes.geometric,shapes.symbols,shapes.geometric}
\usetikzlibrary{shapes.gates.logic.US,shapes.gates.logic.IEC}
\usetikzlibrary{patterns}
\usetikzlibrary{shadows}
\newcommand{\tikzmark}[1]{\tikz[remember picture,overlay,baseline=-.7ex]\node(#1){};}
\input{../Mod_base/rifIndici}
\NeedsTeXFormat{LaTeX2e}
\ProvidesPackage{stand_class}[2019/07/09]

%\RequirePackage[subpreambles=true,sort=true,print=true]{standalone}
\RequirePackage{standalone}
%\standaloneconfig{ mode=image|tex }
\standaloneconfig{mode=buildnew}
\newcommand{\tabincludestandalone}[2][]{%
	$\vcenter{\hbox{\includestandalone[#1]{#2}}}$}

\NeedsTeXFormat{LaTeX2e}
\ProvidesPackage{matematica}[2019/07/09]
\usepackage{bm}
\DeclareMathOperator{\arccot}{arccot}
%\DeclareMathOperator{\sec}{sec}
\DeclareMathOperator{\cosec}{cosec}
\DeclareMathOperator{\difs}{\bigtriangleup}
\providecommand{\abs}[1]{\lvert#1\rvert}
\providecommand{\norm}[1]{\lVert#1\rVert}
%operatore derivata
\newcommand{\OpD}[1]{D\left[#1\right]} 
%\newcommand{\gradi}{\ensuremath{^\circ}}
%\newcommand{\gradi}{\ensuremath{\degree}}
%\newcommand{\minuti}{\ensuremath{\arcminute}}
\newcommand{\PA}[1]{\boldsymbol{\mathcal{#1}}}
\newcommand{\PDA}{\PA{P}}
\newcommand{\numberset}[1]{\mathbb{#1}}

\newcommand{\Ni}{\numberset{N}}
\newcommand{\Nz}{\numberset{N}^{}_{0}}
\newcommand{\Z}{\numberset{Z}}
\newcommand{\Zn}{\numberset{Z^-}}
\newcommand{\Zp}{\numberset{Z^{+}}}
\newcommand{\Q}{\numberset{Q}}
\newcommand{\Qp}{\numberset{Q^{+}}}
\newcommand{\Qn}{\numberset{Q^{-}}}
\newcommand{\R}{\numberset{R}}
\newcommand{\Rpos}{\numberset{R^{+}}}
\newcommand{\Rneg}{\numberset{R^{-}}}
\newcommand{\nq}{\ensuremath{\mathbb{Q}}}
\newcommand{\bld}[1]{\mbox{\boldmath $#1$}}
\newcommand{\Co}{\numberset{C}}
\newcommand{\function}[5]{%
	\begin{array}{@{}r<{{}}@{}c@{}c@{}l@{}}
		#1\colon & #2 & {}\to{}     & #3 \\
		& #4 & {}\mapsto{} & #5
\end{array}}
\newcommand{\funzione}[3]{#1\colon #2\to{}#3}
%spazi fantasma
%\newcommand{\spa}{\phantom{1}}
%\let\origcleardoublepage\cleardoublepage
%\newcommand{\clearemptydoublepage}{%
%\clearpage
%{\pagestyle{empty}\origcleardoublepage}%
%}
%\let\cleardoublepage\clearemptydoublepage
%spazio insecabile
%\newcommand{\nbs}{\nobreakspace}
%le costanti
\newcommand{\costante}{\textrm{costante}}
%\DeclareMathOperator{\uimm}{j}
% The number `e'
\providecommand*{\eu}%
{\ensuremath{\mathrm{e}}}
\providecommand*{\uimm}%
%{\ensuremath{\mathrm{j}}}
{j\mkern1mu}
% parte immaginaria di 1: \parteimm[\dot{x}(t)]
% \DeclareMathOperator{\Re}{Re}
% \DeclareMathOperator{\Im}{Im}
% \DeclareMathOperator{\Arg}{Arg}
%coniugio
\newcommand{\conj}[2][3]{{}\mkern#1mu\overline{\mkern-#1mu#2}}
\let\misura\conj
%  \renewoperator{\Re}%
%  {\mathrm{Re}}{\nolimits}
%  \renewoperator{\Im}%
%  {\mathrm{Im}}{\nolimits}
\DeclareMathOperator{\mcd}{mcd}
\DeclareMathOperator{\mcm}{mcm}
% % % %prodotto in croce
\newcommand{\prodcroce}[4]{%
	\begin{tikzpicture}[thick]
	\def\x{2.8mm}
	\def\h{2.4mm}
	\def\dist{12mm}%1cm
	\node at (0,0) {$\displaystyle \frac{#1}{#2}$};
	\node at (\dist,0) {$\displaystyle \frac{#3}{#4}$};
	\node at (2.5*\dist,0) {$#1\cdot #4=#2\cdot #3$};
	% collegamento termini
	\draw[-stealth] (\x, \h)--(\dist-\x,-\h); 
	\draw[-stealth] (\x,-\h)--(\dist-\x, \h);
	\end{tikzpicture}%
}
\newcommand*{\rectangolo}[1]{\tikz[baseline=(char.base)]{
		\node[shape=rectangle,draw,inner sep=2pt] (char) {#1};}}
\newcommand*{\circled}[1]{\tikz[baseline=(char.base)]{
		\node[shape=circle,draw,inner sep=2pt] (char) {#1};}}%numeri in circonferenze
\newcommand*{\numcircledmod}[1]{\raisebox{.5pt}{\textcircled{\raisebox{-.9pt} {#1}}}}%numeri in circonferenze 2 versione 
\newcommand*{\csqrt}[2]{\sqrt[\numcircledmod{#1}]{#2}}
\DeclareMathSymbol{\vs}{\mathpunct}{letters}{"3B}
%\newcommand{\matsig}[1]{\boldsymbol{\mathcal{#1}}}
\newcommand{\coord}[2]{\left(#1 \vs #2\right)}
\newcommand{\xunodue}[3]{\dfrac{-(#2)\pm\sqrt{(#2)^2-4(#1)(#3)}}{2(#1)}}
\usepackage{cancel}
% % % % % % % % % % insiemi
\usepackage{braket}
\usepackage{xlop}
  \input{../Mod_base/macro_xlop} % scomposizione in fattori primi e operazioni

  \getprime{20}%

\NeedsTeXFormat{LaTeX2e}
\ProvidesPackage{tabelle}[2019/07/09]
\RequirePackage{booktabs}
\RequirePackage{subcaption}
\captionsetup{tableposition=top,figureposition=bottom,font=small,hypcap=false}
\RequirePackage{array}
\RequirePackage{float}
\RequirePackage{rotating}
\RequirePackage{color}
\RequirePackage{colortbl}
\RequirePackage{multirow}
\RequirePackage{wrapfig}
\RequirePackage{longtable}
%\tikzset{every picture/.style={scale=0.3,every picture/.style={}}}
\tikzset{main base/.style={draw,thick,text centered, }, }
\tikzset{main node/.style={rectangle, draw,rounded corners,main base}, }
%      
%\tikzset{main verb/.style={minimum size=1cm}, }
%\tikzset{linea/.style={-triangle 90}, }
%\tikzset{main node2/.style={rectangle, draw,thick,
%    text width=7em, text centered, rounded corners, minimum height=3em}, }
\tikzset{main node2/.style={main node,
    text width=7em, minimum height=3em}, }
\tikzset{main verb/.style={minimum size=1cm}, }
\tikzset{linea/.style={-triangle 90,thick,draw}}
%\tikzset{linea/.style={-triangle 90,thick,draw}}
\tikzset{linea2/.style={-triangle 90,draw}}
\tikzset{primo/.style={circle,draw,inner sep=0pt,minimum size=1pt,thick}, }
\tikzset{
    start-end/.style={
        draw,
        rectangle,
        rounded corners,main base,
    } ,
    input/.style={ % requires library shapes.geometric
        draw,
        trapezium,
        trapezium left angle=60,
        trapezium right angle=120,main base
    },
    operation/.style={
        draw,thick,
        rectangle,main base,
    },
    loop/.style={ % requires library shapes.misc
        draw,
        chamfered rectangle,
        chamfered rectangle xsep=2cm
    },
    decision/.style={ % requires library shapes.geometric
        draw,
        diamond,
        aspect=#1,main base
    },
    decision/.default=1,
    print/.style={ % requires library shapes.symbols
        draw,
        tape,
        tape bend top=none
    },
    connection/.style={
        draw,
        circle,
        radius=5pt,
    },
    process rectangle outer width/.initial=0.15cm,
    predefined process/.style={
        rectangle,
        draw,
        append after command={
        \pgfextra{
          \draw
          ($(\tikzlastnode.north west)-(0,0.5\pgflinewidth)$)--
          ($(\tikzlastnode.north west)-(\pgfkeysvalueof{/tikz/process rectangle outer width},0.5\pgflinewidth)$)--
          ($(\tikzlastnode.south west)+(-\pgfkeysvalueof{/tikz/process rectangle outer width},+0.5\pgflinewidth)$)--
          ($(\tikzlastnode.south west)+(0,0.5\pgflinewidth)$);
          \draw
          ($(\tikzlastnode.north east)-(0,0.5\pgflinewidth)$)--
          ($(\tikzlastnode.north east)+(\pgfkeysvalueof{/tikz/process rectangle outer width},-0.5\pgflinewidth)$)--
          ($(\tikzlastnode.south east)+(\pgfkeysvalueof{/tikz/process rectangle outer width},0.5\pgflinewidth)$)--
          ($(\tikzlastnode.south east)+(0,0.5\pgflinewidth)$);
        }  
        },
        text width=#1,
        align=center
    },
    predefined process/.default=1.75cm,
    man op/.style={ % requires library shapes.geometric
        draw,
        trapezium,
        shape border rotate=180,
        text width=2cm,
        align=center,
    },
    extract/.style={
        draw,
        isosceles triangle,
        isosceles triangle apex angle=60,
        shape border rotate=90
    },
    merge/.style={
        draw,
        isosceles triangle,
        isosceles triangle apex angle=60,
        shape border rotate=-90
    },
}
\DeclareCaptionFormat{grafico}{\textbf{Grafico \thefigure}#2#3}
\DeclareCaptionFormat{esempio}{\textbf{Esempio \thefigure}#2#3}
\newcolumntype{L}{>{$\displaystyle}l<{$}}
\newcolumntype{C}{>{$\displaystyle}c<{$}}
\newcolumntype{R}{>{$\displaystyle}r<{$}}
\newcolumntype{W}{>{\sffamily\Large $}c<{$}}
\makeindex[options=-s ../Mod_base/oldclaudio.sti]
\NeedsTeXFormat{LaTeX2e}
\ProvidesPackage{pagina}[2019/07/09]
\RequirePackage{indentfirst}
\RequirePackage{fancyhdr}
\pagestyle{fancy}
\fancyhead{}
\fancyhead[L]{\ifthenelse{\isodd{\value{page}}}{\leftmark}{\rightmark}}
\fancyhead[R]{\thepage}
\renewcommand{\headrulewidth}{0pt}
\fancyfoot{}
\fancyfoot[L]{\bfseries\today}
\fancyfoot[R]{\bfseries\DTMcurrenttime}

\NeedsTeXFormat{LaTeX2e}
\ProvidesPackage{date}[2019/07/09]
% % % % % % % % % % % % % % % % % % % % % % % % % % % % % % % % % % % % % %
\RequirePackage[useregional,showdow]{datetime2}
% % % % % % % % % % % % % % % % % % % % % % % % % % % % % % % % % % % % % %

\input{../Mod_base/loghi}
\input{../Mod_base/unita_misura}
\input{../Mod_base/tcolorboxgest}
\input{../Mod_base/glossario}
\newglossary[slg]{symbolslist}{sym}{sbl}{Elenco di simboli}

\makeglossaries
\loadglsentries{glossario/glossari4}
\loadglsentries[altacronym]{glossario/acronimi1}
\loadglsentries[acronym]{glossario/simboli1}

\newcommand{\tabincludestandalone}[2][]{%
	$\vcenter{\hbox{\includestandalone[#1]{#2}}}$}

\newcommand{\HRule}{\rule{\linewidth}{0.5mm}}
\makeatletter
\renewcommand\frontmatter{%
	\cleardoublepage
	\@mainmatterfalse
	\pagenumbering{arabic}}
\renewcommand\mainmatter{%
	\cleardoublepage
	\@mainmattertrue}
\makeatother
\input{../Mod_base/indice}
\input{../Mod_base/utili}
\usepackage{hyperxmp}
\usepackage{hyperref}
\title{Appunti di Matematica}
\author{Claudio Duchi}
\date{\datetime}
\usepackage{tkz-berge}
\newcommand{\polylogo}[1]{
\begin{center}
			\begin{tikzpicture}
	\renewcommand*{\VertexBallColor}{green!50!black}
	\GraphInit[vstyle=Art]
	\grComplete[RA=6]{#1}
	\end{tikzpicture}
\end{center}
}
\hypersetup{%
	pdfencoding=auto,
	urlcolor={blue},
	pdftitle={Appunti di matematica },
	pdfsubject={Appunti quinto},
	pdfstartview={FitH},
	pdfpagemode={UseOutlines},
	pdflicenseurl={http://creativecommons.org/licenses/by-nc-nd/3.0/},
	pdflang={it},
	pdfmetalang={it},
	pdfkeywords={Analisi},
	pdfcopyright={Copyright (C) 2017, Claudio Duchi},
	pdfcontacturl={http://breviariomatematico.altervista.org},
	pdfcontactpostcode={},
	pdfcontactphone={},
	pdfcontactemail={claduc},
	pdfcontactcountry={Italy},
	pdfcontactcity={Perugia},
	pdfcontactaddress={},
	pdfcaptionwriter={Claudio Duchi},
	pdfauthortitle={},%
	pdfauthor={Claudio Duchi},
	linkcolor={blue},
	colorlinks=true,
	citecolor={red},
	breaklinks,
	bookmarksopen,
	verbose,
	baseurl={http://breviariomatematico.altervista.org}
}
\includeonly{limiti,
limfunzionirazionali,
asintoti}
\listfiles
\begin{document}
\frontmatter
\begin{titlepage}
	
	\begin{center}
		
		
		% Upper part of the page
			%\includestandalone{../Mod_base/Lgrande2}\\[1cm]    
			
		\includegraphics{../Mod_base/Lgrande2}\\[1cm]    
		\textsc{\LARGE Claudio Duchi}\\[1.5cm]
		
		%\textsc{\Large Final year project}\\[0.5cm]
		
		
		% Title
		\HRule \\[0.4cm]
		{ \huge \bfseries Appunti di matematica}\\[0.4cm]
		{\bfseries QUINTO}\\[0.4cm]
		\vfill
				% Bottom of the page
			\polylogo{15}		
		{\large $-$\DTMnow$-$}
		\end{center}

	\end{titlepage}
\setcounter{page}{2}
\hypersetup{pageanchor=true}
\input{../Mod_base/copyright}
\tableofcontents 
\cleardoublepage
\listoftables
\addcontentsline{toc}{chapter}{\listtablename}
\cleardoublepage
\listoffigures
\addcontentsline{toc}{chapter}{\listfigurename}
\cleardoublepage\renewcommand\lstlistlistingname{Elenco esempi}
\addcontentsline{toc}{chapter}{\lstlistlistingname}
\addcontentsline{toc}{section}{Esempi}
\lstlistoflistings{}
\tcblistof[\section*]{thm}{Esempi}
\addcontentsline{toc}{section}{Contro esempi}
\tcblistof[\section*]{cthm}{Contro esempi}
 \listoftodos
\mainmatter%
\chapter{Disequazioni}
 \section{Diseguaglianze}
\label{sec:Disequglianze}
Iniziamo con un po' di vocabolario. La tabella~\vref{tab:disuguaglianze} mostra le possibili disuguaglianze\index{Disuguaglianza} e il modo corretto di  leggerle.
\begin{table}
	\centering
	\begin{tabular}{lcll}
		\toprule
		<&$a<b$&minore stretto&<<a è minore di b>>\\
		>&$a>b$&maggiore stretto& <<a è maggiore di b>>\\
		$\leq$&$a\leq b$&minore o uguale& <<a è minore di b>> o <<a è uguale a b>> \\
		$\geq$&$a\geq b$&maggiore o uguale&<<a è maggiore di b>> o <<a è uguale a b>>\\
		\bottomrule
	\end{tabular}
	\caption{Disuguaglianze}
	\label{tab:disuguaglianze}
\end{table}

\section{Disequazioni di primo grado}
\label{sec:Disequuazionidiprimogrado}
\begin{definizionet}{}{}
	Una disequazione\index{Disequazione} è una diseguaglianza\index{Disuguaglianza} in cui compare un'incognita.
\end{definizionet}
\begin{definizionet}{Forma normale}{}
	Una disequazione di primo grado è in forma normale\index{Disequazione!forma normale} se è scritta in una di queste forme
	\begin{equation}
	ax\left\{ \begin{aligned}
	<b\\
	\leq b\\
	\geq b\\
	>b
	\end{aligned}\right .   
	\end{equation}
\end{definizionet}
\subsection{Risolvere una disequazione di primo grado}
Per risolvere una disequazione bisogna avere chiaro cosa si intende per soluzione\index{Disequazione!soluzione}
\begin{definizionet}{Soluzione}{}
	Una soluzione\index{Disequazione!soluzione} per una disequazione è un valore che sostituito all'incognita rende vera la disuguaglianza
\end{definizionet}

La definizione sembra simile a quella per le equazioni. Per un'equazione abbiamo: <<una soluzione è quel valore che rende vera l'uguaglianza>>\par
La somiglianza è solo apparente, infatti per un'equazione di primo grado in un incognita, la soluzione è un valore, per una disequazione la soluzione è un intervallo. Per esempio la disequazione elementare$X>1$ ha per soluzione tutti i numeri che sono maggiori di uno cioè l'intervallo $]1 +\infty [$.\par
Il metodo per risolvere una disequazione di primo è simile a quello per utilizzato con una equazione di pari grado, cioè la separazione delle variabili\index{Separazione!variabili}.\par Un esempio è il seguente. Supponiamo di dover risolvere 
\begin{esempiot}{Disequazione di primo grado}{}
	\begin{equation}
	3x+5<2x+6\label{equ:PrimoGradoDisequazione1}
	\end{equation}
\end{esempiot}
procediamo come nella figura\nobs\vref{fig:esempioDisequazioniPgrado1}
\begin{figure}
	\begin{NodesList}[margin=4.5cm]
		\centering
		\begin{align*}
		3x+5<&2x+6\AddNode\\[.5cm] 
		3x+5-2x<&6\AddNode\\[.5cm] %\AddNode[2]\\ 
		3x-2x<&6-5\AddNode\\
		x<&1\AddNode
		\end{align*}
			\tikzset{LabelStyle/.style = {right=0.1cm,pos=0.5,text=red,fill=white}}
		\LinkNodes{Sposto $2x$ e cambio di segno}
		\LinkNodes{Sposto $+5$ e cambio di segno}%
		\LinkNodes{Sommo}%
	\end{NodesList}
	\captionsetup{format=esempio,list=no}
	\caption{Risoluzione disequazione\nobs\vref{equ:PrimoGradoDisequazione1}}
	\label{fig:esempioDisequazioniPgrado1}
\end{figure}

Il procedimento è  quello della risoluzione di un'equazione di primo grado, si trasportano a sinistra i valori con l'incognita, a destra i numeri, vale la stessa regola che si usa per le equazioni: spostando i termini rispettto al verso, si cambia di segno. Per rappresentare la soluzione si usa un metodo grafico che rappresenta le soluzioni. Il grafico dell'esempio è la figura\nobs\vref{fig:esempioDisequazioniPgradografico1}. Per disegnare il grafico della soluzione si procede in questa maniera: 
\begin{procedurat}{}{}
	\begin{enumerate}
		\item Si traccia una linea orizzontale orientata, l'asse dei numeri.
		\item Si mette sotto di essa la soluzione trovata.
		\item In corrispondenza della soluzione si traccia un segmento verticale.
		\item Si guarda la soluzione e dalla parte superiore del segmento si traccia una semiretta continua nella direzione della freccia e una semiretta tratteggiata dal lato opposto.
	\end{enumerate}
\end{procedurat}
\begin{figure}
	\centering
	\begin{tikzpicture}
	\draw[ -triangle 90](0,0)--(5,0);
	\draw(2,0)--(2,1);
	%%%%%soluzioni
	%%%sinistra	
	\draw[dashed](2,1)--(5,1);
	%%destra
	\draw(2,1)--(0,1);
	\node at (2,-0.5) {1};
	\end{tikzpicture}
	\captionsetup{format=grafico,list=no}
	\caption[]{Disequazione\nobs\vref{equ:PrimoGradoDisequazione1}}
	\label{fig:esempioDisequazioniPgradografico1}
\end{figure}
\begin{esempiot}{Disequazione di primo grado, cambio di verso}{}
	\begin{equation}
	20x-12\geq 5x+8\label{equ:PrimoGradoDisequazione2}
	\end{equation}
\end{esempiot}
procediamo come nella figura\nobs\vref{fig:esempioDisequazioniPgrado2}
\begin{figure}
	\begin{NodesList}[margin=4.5cm]
		\centering
		\begin{align*}
		2x-12\geq 5x+8\AddNode\\[.5cm] 
		2x-12-5x\geq+8\AddNode\\[.5cm] %\AddNode[2]\\ 
		2x-5x\geq 12+8\AddNode\\
		-3x\geq &20\AddNode\\
		x\leq&-\dfrac{20}{3}\AddNode
		\end{align*}
		\tikzset{LabelStyle/.style = {right=0.1cm,pos=0.5,text=red,fill=white}}
		\LinkNodes{Sposto $5x$ e cambio di segno}
		\LinkNodes{Sposto $+12$ e cambio di segno}%
		\LinkNodes{Sommo}%
		\LinkNodes{Cambio di segno e di verso}%
	\end{NodesList}
	\captionsetup{format=esempio,list=no}
	\caption{Risoluzione disequazione\nobs\vref{equ:PrimoGradoDisequazione1}}
	\label{fig:esempioDisequazioniPgrado2}
\end{figure}
\section{Disequazioni di secondo grado}
\section{Disequazioni fratte}
\section{Disequazioni di grado superiore al secondo}
\section{Sistemi di disequazioni intere e fratte}
\chapter{Funzioni}
\section{Definizione}
\section{Classificazione}
\section{Campo di esistenza}
\section{Segno di una funzione}
\section{Intersezione con gli assi}
\chapter{Limiti}
\section{Limite finito per x che tende a valore finito}
Supponiamo di avere una funzione $\funzione{f}{\R}{\R}$ definita \[f(x)=3x+1\] Costruiamo la seguente tabella della funzione per valori di $x$ prossimi ad uno.
\begin{center}
	
\begin{tabular}{@{}Lccccccc}
	\toprule
&&	\multicolumn{2}{R}{\longrightarrow}&&\multicolumn{2}{L}{\longleftarrow}\\
x	& 0.9 & 0.99 & 0.999&1 &1.001  &1.01  & 1.1 \\
\cmidrule(r){2-8} 
f(x)	&3.7  & 3.97 &3.997 & & 4.003 & 4.03 & 4.3 \\ 
\cmidrule(r){2-8} 
\abs{f(x)-4}& 0.3 &0.03  &0.003  &  & 0.003 &0.03& 0.3 \\ 
\bottomrule
\end{tabular}\captionof{table}{Limite finito per x che tende a valori finito}
\end{center}
Guardando la tabella, si nota che per valori di $x$ prossimi a uno la funzione assume valori prossimi al quattro. Dalla tabella risulta che la differenza tra il valore della funzione e il quattro può essere resa piccola a piacere. Quanto più piccola si vuole la differenza tra la funzione e il numero esiste un intorno del punto dell'uno per cui preso un valore appartenente a questo intorno, la differenza è minore di quanto fissato. Se succede questo si dice che \[\lim_{x\to 1}3x+1=4 \]
In generale \[ \forall \epsilon>0\; \exists\; \delta_\epsilon>0 \; |\; \forall x \; 0<\abs{x-x_0}<\delta \Rightarrow \abs{f(x)-l}<\epsilon \]
 \[\lim_{x\to x_0}f(x)=l \]
 \section{Limite sinistro e limite destro}
 Per spiegare il concetto usiamo la seguente funzione\index{Funzione!segno} segno\[f(x)=\dfrac{\abs{x}}{x}\]Questa funzione vale \[ f(x)=\begin{cases}
 +1&x>0\\
 -1&x<0\\
 \text{Indefinita}&x=0
 \end{cases}\]
\chapter{Limiti di funzioni razionali}
\section{Forma indeterminata infinito su infinito}
Supponiamo di avere una funzione razionale fratta
\begin{esempiot}{Forma indeterminata infinito su infinito}{a}
	\[f(x)=\dfrac{x^2+2x+1}{2x^3+3x+1} \]
\end{esempiot}
Il limite
\[\lim_{x\to \infty}\dfrac{x^2+2x+1}{2x^3+3x+1}=\dfrac{\infty}{\infty}\]
è una forma indeterminata che risolvo nel seguente modo
\begin{align*}\index{Forma!indeterminata!infinito su infinito}
\lim_{x\to \infty}\dfrac{x^2+2x+1}{2x^3+3x+1}=&\\
=&\lim_{x\to \infty}\dfrac{x^2(1+\dfrac{2}{x}+\dfrac{1}{x^2})}{x^3(2+\dfrac{3}{x^2}+\dfrac{1}{x^3})}
\intertext{I termini dentro alla parentesi al numeratore valgono uno mentre i termini al denominatore valgono due ottengo}
=&\lim_{x\to \infty}\dfrac{x^2\cdot 1}{x^3\cdot 2}\\
=&\lim_{x\to \infty}\dfrac{1}{2}\cdot\dfrac{1}{x}=0
\end{align*}
quindi la forma indeterminata è risolta.
\section{Forma indeterminata zero su zero}
Consideriamo la funzione 
\begin{esempiot}{Forma indeterminata zero su zero}{b}
\[f(x)=\dfrac{3x^2-4x-4}{5x^2-8x-4} \]
\end{esempiot}
se calcoliamo il limite\[\lim_{x\to 2}\dfrac{3x^2-4x-4}{5x^2-8x-4} \]
\begin{align*}\index{Forma!indeterminata!zero su zero}
\lim_{x\to 2}\dfrac{3x^2-4x-4}{5x^2-8x-4}=&\\
=&\lim_{x\to 2}\dfrac{12-8-4}{20-16-4}\\
=&\dfrac{0}{0}\\
\end{align*}
Dato che il numeratore e  denominatore sono due polinomi di secondo grado possiamo utilizzare 
\[ax^2+bx+c=a(x-x_1)(x-x_2) \] dove $x_1$ e $x_2$ solo le soluzioni di \[ ax^2+bx+c=0\]
quindi \begin{align*}
3x^2-4x-4=&0\\
x_{1,2}=&\dfrac{4\pm\sqrt{64}}{6}\\
=&\begin{cases}
x_1=\dfrac{4+8}{6}=2\\
x_2=\dfrac{4-8}{6}=-\dfrac{2}{3}
\end{cases}
\intertext{quindi}
3x^2-4x-4=&3(x-2)(x+\dfrac{2}{3})
\intertext{resta}
5x^2-8x-4=&0\\
x_{1,2}=&\dfrac{8\pm\sqrt{144}}{10}\\
=&\begin{cases}
x_1=\dfrac{8+12}{10}=2\\
x_2=\dfrac{8-12}{10}=-\dfrac{2}{5}
\end{cases}
\intertext{quindi}
5x^2-8x-4=&5(x-2)(x+\dfrac{2}{5})
\end{align*}
Utilizzando i risultati precedenti otteniamo
\begin{align*}
\lim_{x\to 2}\dfrac{3x^2-4x-4}{5x^2-8x-4}=&\\
=&\lim_{x\to 2}\dfrac{3(x-2)(x+\dfrac{2}{3})}{5(x-2)(x+\dfrac{2}{5})}\\
\intertext{semplificando}
=&\lim_{x\to 2}\dfrac{3(x+\dfrac{2}{3})}{5(x+\dfrac{2}{5})}\\
=&\lim_{x\to 2}\dfrac{3x+2}{5x+2}=\dfrac{8}{12}=\dfrac{2}{3}\\
\end{align*}
\chapter{Asintoti}
Un asintoto è una retta che, in maniera semplice, a cui si avvicina una curva senza mai toccarla. Vi sono tre tipi di asintoti. 
\section{Asintoti verticali}
Gli asintoti verticali, per una funzione razionale fratta, sono legati ai valori del dominio.
Una funzione ha un asintoto verticale se 
\[\lim_{x\to k}f(x)=\infty\]
L'asintoto ha equazione \[ x=k\]
Normalmente vi sono quattro tipi di  asintoti che mostreremo tramite esempi.
Iniziamo \index{Asintoto!verticale} 
considerando la funzione 
\begin{esempiot}{Asintoto verticale}{ass1}
\[f(x)=\dfrac{x^2-x-6}{x^2+x-6} \]
\end{esempiot}
Determiniamo il dominio\index{Funzione!dominio}
della funzione. In una razionale fratta il dominio si determina verificando quando il denominatore si annulla quindi
\begin{align*}
x^2+x-6=&0\\
\intertext{risolvo l'equazione ed ottengo}
&\begin{cases}
x_1=-3\\
x_2=2
\end{cases}
\end{align*}
Quindi il dominio della funzione è $\R-\lbrace2,-3\rbrace$ Abbiamo due candidati per gli asintoti verticali.
Calcolo il limite \[ \lim_{x\to 2}\dfrac{x^2-x-6}{x^2+x-6}=\dfrac{-4}{0}=\infty\] Valutiamo i limiti destro e sinistro. 
\begin{align*}
\lim_{x\to 2^+}\dfrac{x^2-x-6}{x^2+x-6}=&\\
\intertext{dato che}
4^+-2^+-6<&0\\
4^++2^+-6>&0\\
\lim_{x\to 2^+}\dfrac{x^2-x-6}{x^2+x-6}=&-\infty\\
\end{align*}
\begin{align*}
\lim_{x\to 2^-}\dfrac{x^2-x-6}{x^2+x-6}=&\\
\intertext{dato che}
4^--2^+-6<&0\\
4^-+2^--6<&0\\
\lim_{x\to 2^-}\dfrac{x^2-x-6}{x^2+x-6}=&+\infty\\
\end{align*}
Quindi \[x=2\] è un asintoto verticale.
L'asintoto è del tipo più infinito meno infinito come il grafico~\vref{fig:asintotopm}
\begin{figure}
	\centering
\includestandalone{grafici/asintotoPM}
\captionsetup{format=grafico}
	\caption[Asintoto piu infinito meno infinito]{Asintoto piu infinito meno infinito}
	\label{fig:asintotopm}
\end{figure}

Calcolo il limite \[\lim_{x\to -3}\dfrac{x^2-x-6}{x^2+x-6}=\dfrac{6}{0}=\infty\] Valutiamo i limiti destro e sinistro. 
\begin{align*}
\lim_{x\to -3^+}\dfrac{x^2-x-6}{x^2+x-6}=&\\
\intertext{dato che}
(-3^+)^2<&9\\
(-3^+)^2-(-3^+)-6>&0\\
(-3^+)^2+(-3^+)-6<&0\\
\lim_{x\to -3^+}\dfrac{x^2-x-6}{x^2+x-6}=&-\infty\\
\end{align*}
\begin{align*}
\lim_{x\to -3^-}\dfrac{x^2-x-6}{x^2+x-6}=&\\
\intertext{dato che}
(-3^-)^2>&9\\
(-3^-)^2-(-3^-)-6>&0\\
(-3^-)^2+(-3^-)-6>&0\\
\lim_{x\to -3^-}\dfrac{x^2-x-6}{x^2+x-6}=&+\infty\\
\end{align*}
Quindi \[x=-3\] è un asintoto verticale.
\begin{esempiot}{Asintoto verticale}{AssMP}
	\[f(x)=\dfrac{2-x}{x-1}\]
\end{esempiot}
Dato che la funzione è una razionale fratta ne calcolo il dominio ponendo il denominatore uguale a zero\[x-1=0\]Ha soluzione $x=1$ quindi il domino è $\R-\lbrace1\rbrace$

Calcoliamo il limite\[\lim_{x\to 1}\dfrac{2-x}{x-1}=\dfrac{2-1}{0}=\infty\]
Valutiamo i limiti destro e sinistro
\begin{align*}
\lim_{x\to 1^+}\dfrac{2-x}{x-1}=&\\
\intertext{dato che}
2-1^+>&0\\
1^+-1>&0\\
\intertext{avremo che}
\lim_{x\to 1^+}\dfrac{2-x}{x-1}=&+\infty\\
\lim_{x\to 1^-}\dfrac{2-x}{x-1}=&\\
\intertext{dato che}
2-1^->&0\\
1^--1<&0\\
\intertext{avremo che}
\lim_{x\to 1^-}\dfrac{2-x}{x-1}=&-\infty\\
\end{align*}
Quindi \[x=2\] è un asintoto verticale.
L'asintoto è del tipo meno infinito più infinito come il grafico~\vref{fig:asintotomp}
\begin{figure}
	\centering
	\includestandalone{grafici/asintotoMP}
	\captionsetup{format=grafico}
	\caption[Asintoto meno infinito più infinito]{Asintoto meno infinito più infinito}
	\label{fig:asintotomp}
\end{figure}
Consideriamo un altra  funzione
\begin{esempiot}{Asintoto verticale}{AsintotoPP}
	\[f(x)=\dfrac{x^2+1}{x^2-2x-1} \]
\end{esempiot}\index{Asintoto!verticale}
La funzione è una razionale fratta per determinarne il dominio pongo il denominatore uguale a zero e risolvo
\[x^2-2x-1=0\]Ha soluzione $x=1$ quindi il domino è $\R-\lbrace1\rbrace$

Calcoliamo il limite\[\lim_{x\to 1}\dfrac{x^2+1}{x^2-2x-1}=\dfrac{2}{0}=\infty\]
Valutiamo i limiti destro e sinistro
\begin{align*}
\lim_{x\to 1^+}\dfrac{x^2+1}{x^2-2x-1}=&\\
\intertext{dato che}
(1^+)^2+1>&0\\
(1^+)^2 -2\cdot 1^+ +1>&0\\
\intertext{quindi}
\lim_{x\to 1^+}\dfrac{x^2+1}{x^2-2x-1}=&+\infty\\
\intertext{Valuto ora}
\lim_{x\to 1^-}\dfrac{x^2+1}{x^2-2x-1}=&\\
\intertext{dato che}
(1^+)^2+1>&0\\
(1^+)^2 -2\cdot 1^+ +1>&0\\
\intertext{avremo che}
\lim_{x\to 1^-}\dfrac{x^2+1}{x^2-2x-1}=&+\infty\\
\end{align*}
Quindi \[x=1\] è un asintoto verticale.
L'asintoto è del tipo più infinito più infinito come il grafico~\vref{fig:asintotopp}
\begin{figure}
	\centering
	\includestandalone{grafici/asintotoPP}
	\captionsetup{format=grafico}
	\caption[Asintoto più infinito più infinito]{Asintoto piu infinito più infinito}
	\label{fig:asintotopp}
\end{figure}

Consideriamo un altra  funzione
\begin{esempiot}{Asintoto verticale}{AsintotoMM}
	\[f(x)=\dfrac{x-1}{x^2} \]
\end{esempiot}\index{Asintoto!verticale}
La funzione è una razionale fratta per determinarne il dominio pongo il denominatore uguale a zero e risolvo
\[x^2=0\]Ha soluzione $x=0$ quindi il domino è $\R-\lbrace0\rbrace$

Calcoliamo il limite\[\lim_{x\to 0}\dfrac{x-1}{x^2}=\dfrac{-1}{0}=\infty\]
Valutiamo i limiti destro e sinistro
\begin{align*}
\lim_{x\to 0^+}\dfrac{x-1}{x^2}=&\\
\intertext{dato che}
(0^+)^2>&0\\
0^+ -1<&0\\
\intertext{quindi}
\lim_{x\to 0^+}\dfrac{x-1}{x^2}=&-\infty\\
\intertext{Valuto ora}
\lim_{x\to 0^-}\dfrac{x-1}{x^2}=&\\
\intertext{dato che}
(0^-)^2>&0\\
0^- -1<&0\\
\intertext{avremo che}
\lim_{x\to 0^-}\dfrac{x-1}{x^2}=&-\infty\\
\end{align*}
Quindi \[x=0\] è un asintoto verticale.
L'asintoto è del tipo meno infinito meno infinito come il grafico~\vref{fig:asintotomm}
\begin{figure}
	\centering
	\includestandalone{grafici/asintotomm}
	\captionsetup{format=grafico}
	\caption[Asintoto meno infinito meno infinito]{Asintoto meno infinito meno infinito}
	\label{fig:asintotomm}
\end{figure}

Non tutte le funzioni razionali fratte hanno asintoti verticali
\begin{cesempiot}{Funzione senza asintoti verticali}{ass2}
	\[f(x)=\dfrac{x^2+3x}{x^2-x+1} \]
\end{cesempiot}
Determino il dominio
\begin{align*}
x^2-x+1=&0\\
\intertext{risolvo l'equazione ed ottengo}
x_{1,2}=\dfrac{1+\pm\sqrt{1-4}}{2}
\end{align*}
L'equazione non ha soluzioni  reali quindi è sempre definita, non ha asintoti verticali.

Negli esempi precedenti gli asintoti verticali, quando esistono, compaiono sempre in coppia tuttavia possiamo avere casi in cui abbiamo un solo asintoto verticale. Consideriamoil seguente esempio
\begin{cesempiot}{Asintoto solo destro}{Asintotosolodestro}
\[ f(x)=\begin{cases}
1-x^3&x\leq 1\\
\dfrac{1}{x-1}&x>1
\end{cases}\]
\end{cesempiot}
\begin{figure}
	\centering
	\includestandalone{grafici/asintotosolodestro}
	\captionsetup{format=grafico}
	\caption[Asintoto solo destro]{Asintoto solo destro}
	\label{fig:asintotosolodestro}
\end{figure}

Questa funzione di cui il grafico~\vref{fig:asintotosolodestro} non è definita per $x=1$ quindi il dominio è $\R-\lbrace0\rbrace$ abbiamo
\begin{align*}
\lim_{x\to 1^-}f(x)=&0\\
\lim_{x\to 1^+}f(x)=&+\infty\\
\end{align*}
Quindi dato che il limite sinistro è finito mentre il destro è infinito, la funzione ha un solo un limite destro. 
\section{Asintoti orizzontali}
Una funzione $f(x)$ ha un asintoto orizzontale se è verificato che \[\lim_{x\to \infty}f(x)=l \]
Iniziamo con un esempio di funzione che ha un asintoto orizzontale
\begin{esempiot}{Asintoti orizzontali}{ass3}
	\[f(x)=\dfrac{3x^4+2x^3+1}{2x^4+2x+1}\]
\end{esempiot}\index{Asintoto!orizzontale}
Calcolo il limite
\begin{align*}\index{Forma!indeterminata!infinito su infinito}
\lim_{x\to \infty}\dfrac{3x^4+2x^3+1}{2x^4+2x+1}=&\dfrac{\infty}{\infty}\\
\intertext{ottengo una forma indeterminata infinto su infinito quindi, raccogliendo i termini di grado maggiore a numeratore e denominatore}
=&\lim_{x\to \infty}\dfrac{x^4\left(3+\dfrac{2x^2}{x^4}+\dfrac{1}{x^4}\right)}{x^4\left(2+\dfrac{2x}{x^4}+\dfrac{1}{x^4}\right)}\\
\intertext{semplificando}
=&\lim_{x\to \infty}\dfrac{x^4\left(3+\dfrac{2x}{x^2}+\dfrac{1}{x^4}\right)}{x^4\left(2+\dfrac{2}{x^3}+\dfrac{1}{x^4}\right)}\\
\intertext{dato che}
\lim_{x\to \infty}\dfrac{2}{x^2}=&0\\
\lim_{x\to \infty}\dfrac{1}{x^4}=&0\\
\lim_{x\to \infty}\dfrac{2}{x^3}=&0\\
\dfrac{x^4}{x^4}=&1\\
\intertext{ottengo}
=&\lim_{x\to \infty}\dfrac{3}{2}=\dfrac{3}{2}\\
\end{align*}
Abbiamo un limite, e di conseguenza un asintoto orizzontale\[y=\dfrac{3}{2}\]  che si comporta come nel grafico~\vref{fig:asintotorizzo}
\begin{figure}
	\centering
	\includestandalone{grafici/asintotorizzo}
	\captionsetup{format=grafico}
	\caption[Asintoto orizzontale]{Asintoto orizzontale}
	\label{fig:asintotorizzo}
\end{figure}
\begin{esempiot}{Asintoti orizzontali}{ass4}
	\[f(x)=\dfrac{x^2+x+1}{2x^3+2x+3}\]
\end{esempiot}
Per determinare l'asintoto orizzontale calcolo il limite 
\begin{align*}\index{Forma!indeterminata!infinito su infinito}
\lim_{x\to \infty}\dfrac{x^2+x+1}{2x^3+2x+3}=&\dfrac{\infty}{\infty}\\
\intertext{ottengo una forma indeterminata infinto su infinito quindi, raccogliendo i termini di grado maggiore a numeratore e denominatore}
=&\lim_{x\to \infty}\dfrac{x^2\left(1+\dfrac{x}{x^2}+\dfrac{1}{x^2}\right)}{x^3\left(2+\dfrac{2x}{x^3}+\dfrac{3}{x^3}\right)}\\
\intertext{semplificando}
=&\lim_{x\to \infty}\dfrac{x^2\left(1+\dfrac{1}{x}+\dfrac{1}{x^2}\right)}{x^3\left(2+\dfrac{2}{x^2}+\dfrac{3}{x^3}\right)}\\
\intertext{dato che}
\lim_{x\to \infty}\dfrac{1}{x}=&0\\
\lim_{x\to \infty}\dfrac{1}{x^2}=&0\\
\lim_{x\to \infty}\dfrac{2}{x^2}=&0\\
\lim_{x\to \infty}\dfrac{3}{x^3}=&0\\
\dfrac{x^2}{x^3}=&\dfrac{1}{x}\\
\intertext{ottengo}
=&\lim_{x\to \infty}\dfrac{1}{x}\cdot\dfrac{1}{2}=0\\
\end{align*}
Il limite esiste ed è finito quindi la funzione ha un asintoto orizzontale che è \[y=0\]
\begin{cesempiot}{Funzione senza asintoto orizzontale}{ass5}
	\[f(x)=\dfrac{2x^5+x^2+1}{3x^2+1}\]
\end{cesempiot}
Per determinare l'asintoto orizzontale calcolo il limite 
\begin{align*}\index{Forma!indeterminata!infinito su infinito}
\lim_{x\to \infty}\dfrac{2x^5+x^2+1}{3x^2+1}=&\dfrac{\infty}{\infty}\\
\intertext{ottengo una forma indeterminata infinto su infinito quindi, raccogliendo i termini di grado maggiore a numeratore e denominatore}
=&\lim_{x\to \infty}\dfrac{x^5\left(2+\dfrac{x^2}{x^5}+\dfrac{1}{x^5}\right)}{x^2\left(3+\dfrac{1}{x^2}\right)}\\
\intertext{semplificando}
=&\lim_{x\to \infty}\dfrac{x^5\left(2+\dfrac{1}{x^3}+\dfrac{1}{x^5}\right)}{x^2\left(3+\dfrac{1}{x^2}\right)}\\
\intertext{dato che}
\lim_{x\to \infty}\dfrac{1}{x^3}=&0\\
\lim_{x\to \infty}\dfrac{1}{x^5}=&0\\
\lim_{x\to \infty}\dfrac{1}{x^2}=&0\\
\dfrac{x^5}{x^2}=&x^3\\
\intertext{ottengo}
=&\lim_{x\to \infty}x^3\cdot\dfrac{2}{3}=\infty\\
\end{align*}
Il limite non esiste  finito quindi la funzione non ha un asintoto orizzontale.

I tre precedenti esempi ci permettono di definire una regola per determinare se una funzione abbia o non abbia un asintoto orizzontale. Una funziona razionale fratta ha un asintoto orizzontale se il grado del denominatore è maggiore o uguale al grado del numeratore. 
\section{Asintoti obliqui}
Uan funzione $f(x)$ ha un asintoto obliquo se esiste una retta $y=mx+q$ che verifica \[\lim_{x\to \infty}[f(x)-(mx+q)]=0 \]
all'atto pratico se esistono e sono finiti entrambi i seguenti limiti
\begin{align*}
m=&\lim_{x\to \infty}\dfrac{f(x)}{x}\\
q=&\lim_{x\to \infty}[f(x)-mx]
\end{align*}\index{Asintoto!obliquo}
\begin{esempiot}{Asintoti obliqui}{ass6}
	\[f(x)=\dfrac{2x^2+x+1}{x+1}\]
\end{esempiot}
Per determinare l'asintoto obliquo inizio a calcolare il limite
\begin{align*}
m=&\lim_{x\to \infty}\dfrac{f(x)}{x}\\
=&\lim_{x\to \infty}\dfrac{2x^2+x+1}{x+1}\cdot\dfrac{1}{x}\\
=&\lim_{x\to \infty}\dfrac{2x^2+x+1}{x^2+x}\\
=&\lim_{x\to \infty}\dfrac{x^2\left(2+\dfrac{x}{x^2}+\dfrac{1}{x^2}\right)}{x^2\left(1+\dfrac{x}{x^2}\right)}\\
\intertext{semplifico}
=&\lim_{x\to \infty}\dfrac{x^2\left(2+\dfrac{1}{x}+\dfrac{1}{x^2}\right)}{x^2\left(1+\dfrac{1}{x}\right)}\\
\intertext{dato che}
\dfrac{x^2}{x^2}=&1\\
\lim_{x\to \infty}\dfrac{1}{x^2}=&0\\
\lim_{x\to \infty}\dfrac{1}{x}=&0\\
=&\lim_{x\to \infty}1\cdot 2=2
\end{align*} 
Quindi il coefficiente angolare $m$ è due
\[m=2\]
Resta da determinare il coefficiente $q$
\begin{align*}
q=&\lim_{x\to \infty}[f(x)-mx]\\
=&\lim_{x\to \infty}[\dfrac{2x^2+x+1}{x+1}-2x]
\intertext{sommo}
=&\lim_{x\to \infty}[\dfrac{2x^2+x+1-2x^2-2x}{x+1}]\\
\intertext{semplificando}
=&\lim_{x\to \infty}[\dfrac{-x+1}{x+1}]\\
=&\lim_{x\to \infty}[\dfrac{x\left(-1+\dfrac{1}{x}\right)}{x\left(1+\dfrac{1}{x}\right)}]\\
\intertext{dato che}
\dfrac{x}{x}=&1\\
\lim_{x\to \infty}\dfrac{1}{x}=&0\\
=&\dfrac{x^2}{x^2}=&1\\
=&\lim_{x\to \infty}1\cdot\dfrac{-1}{1}=-1\\
\end{align*}
Quindi l'asintoto obliquo è
\[y=2x-1\]	
\glsaddall
\printglossaries
\addcontentsline{toc}{chapter}{\indexname}
\input{../Mod_base/MezziUsati}
\printindex
\end{document}
