\chapter{Disequazioni}
\section{Disequazioni di primo grado}
 \section{Diseguaglianze}
\label{sec:Disequglianze}
Iniziamo con un po' di vocabolario. La tabella~\vref{tab:disuguaglianze} mostra le possibili disuguaglianze\index{Disuguaglianza} e il modo corretto di  leggerle.
\begin{table}
	\centering
	\begin{tabular}{lcll}
		\toprule
		<&$a<b$&minore stretto&<<a è minore di b>>\\
		>&$a>b$&maggiore stretto& <<a è maggiore di b>>\\
		$\leq$&$a\leq b$&minore o uguale& <<a è minore di b>> o <<a è uguale a b>> \\
		$\geq$&$a\geq b$&maggiore o uguale&<<a è maggiore di b>> o <<a è uguale a b>>\\
		\bottomrule
	\end{tabular}
	\caption{Disuguaglianze}
	\label{tab:disuguaglianze}
\end{table}
\begin{figure}
	\centering
	\begin{tikzpicture}[>=latex',line join=bevel,]
	%%
	\node (1) at (27bp,76.177bp) [draw,ellipse] {Una disequazione};
	\node (3) at (123.25bp,18bp) [draw,ellipse] {Determinata};
	\node (2) at (104.11bp,95.152bp) [draw,draw=none] {è};
	\node (5) at (181.47bp,113.47bp) [draw,ellipse] {Impossibile};
	\node (4) at (86bp,172.48bp) [draw,ellipse] {Sempre verificata};
	\draw [->] (1) ..controls (57.307bp,83.635bp) and (62.217bp,84.843bp)  .. (2);
	\draw [->] (2) ..controls (135.93bp,102.69bp) and (140.95bp,103.88bp)  .. (5);
	\draw [->] (2) ..controls (110.95bp,67.577bp) and (113.8bp,56.091bp)  .. (3);
	\draw [->] (2) ..controls (97.637bp,122.79bp) and (94.941bp,134.3bp)  .. (4);
	\end{tikzpicture}
	\caption{Disequazione e soluzioni}
	\label{fig:DidequazioniEsoluzioni}
\end{figure}
\section{Disequazioni di secondo grado}
\section{Disequazioni fratte}
\section{Disequazioni di grado superiore al secondo}
\section{Sistemi di disequazioni intere e fratte}