\chapter{Disequazioni}
 \section{Diseguaglianze}
\label{sec:Disequglianze}
Iniziamo con un po' di vocabolario. La tabella~\vref{tab:disuguaglianze} mostra le possibili disuguaglianze\index{Disuguaglianza} e il modo corretto di  leggerle.
\begin{table}
	\centering
	\begin{tabular}{lcll}
		\toprule
		<&$a<b$&minore stretto&<<a è minore di b>>\\
		>&$a>b$&maggiore stretto& <<a è maggiore di b>>\\
		$\leq$&$a\leq b$&minore o uguale& <<a è minore di b>> o <<a è uguale a b>> \\
		$\geq$&$a\geq b$&maggiore o uguale&<<a è maggiore di b>> o <<a è uguale a b>>\\
		\bottomrule
	\end{tabular}
	\caption{Disuguaglianze}
	\label{tab:disuguaglianze}
\end{table}

\section{Disequazioni di primo grado}
\label{sec:Disequuazionidiprimogrado}
\begin{definizionet}{}{}
	Una disequazione\index{Disequazione} è una diseguaglianza\index{Disuguaglianza} in cui compare un'incognita.
\end{definizionet}
\begin{definizionet}{Forma normale}{}
	Una disequazione di primo grado è in forma normale\index{Disequazione!forma normale} se è scritta in una di queste forme
	\begin{equation}
	ax\left\{ \begin{aligned}
	<b\\
	\leq b\\
	\geq b\\
	>b
	\end{aligned}\right .   
	\end{equation}
\end{definizionet}
\subsection{Risolvere una disequazione di primo grado}
Per risolvere una disequazione bisogna avere chiaro cosa si intende per soluzione\index{Disequazione!soluzione}
\begin{definizionet}{Soluzione}{}
	Una soluzione\index{Disequazione!soluzione} per una disequazione è un valore che sostituito all'incognita rende vera la disuguaglianza
\end{definizionet}

La definizione sembra simile a quella per le equazioni. Per un'equazione abbiamo: <<una soluzione è quel valore che rende vera l'uguaglianza>>\par
La somiglianza è solo apparente, infatti per un'equazione di primo grado in un incognita, la soluzione è un valore, per una disequazione la soluzione è un intervallo. Per esempio la disequazione elementare$X>1$ ha per soluzione tutti i numeri che sono maggiori di uno cioè l'intervallo $]1 +\infty [$.\par
Il metodo per risolvere una disequazione di primo è simile a quello per utilizzato con una equazione di pari grado, cioè la separazione delle variabili\index{Separazione!variabili}.\par Un esempio è il seguente. Supponiamo di dover risolvere 
\begin{esempiot}{Disequazione di primo grado}{}
	\begin{equation}
	3x+5<2x+6\label{equ:PrimoGradoDisequazione1}
	\end{equation}
\end{esempiot}
procediamo come nella figura\nobs\vref{fig:esempioDisequazioniPgrado1}
\begin{figure}
	\begin{NodesList}
		\centering
		\begin{align*}
		3x+5<&2x+6\AddNode\\[.5cm] 
		3x+5-2x<&6\AddNode\\[.5cm] %\AddNode[2]\\ 
		3x-2x<&6-5\AddNode\\
		x<&1\AddNode
		\end{align*}
			\tikzset{LabelStyle/.style = {right=0.1cm,pos=0.5,text=red,fill=white}}
		\LinkNodes{Sposto $2x$ a sinistra\\ e cambio di segno}
		\LinkNodes{Sposto $+5$ a destra e cambio di segno}%
		\LinkNodes{Sommo}%
	\end{NodesList}
	\captionsetup{format=esempio,list=no}
	\caption{Risoluzione disequazione\nobs\vref{equ:PrimoGradoDisequazione1}}
	\label{fig:esempioDisequazioniPgrado1}
\end{figure}

Il procedimento è  quello della risoluzione di un'equazione di primo grado, si trasportano a sinistra i valori con l'incognita, a destra i numeri, vale la stessa regola che si usa per le equazioni: spostando i termini rispettto al verso, si cambia di segno. Per rappresentare la soluzione si usa un metodo grafico che rappresenta le soluzioni. Il grafico dell'esempio è la figura\nobs\vref{fig:esempioDisequazioniPgradografico1}. Per disegnare il grafico della soluzione si procede in questa maniera: 
\begin{procedurat}{}{}
	\begin{enumerate}
		\item si traccia una linea orizzontale orientata, l'asse dei numeri.
		\item si mette sotto di essa la soluzione trovata.
		\item in corrispondenza della soluzione si traccia un segmento verticale.
		\item  si guarda la soluzione e dalla parte superiore del segmento si traccia una semiretta continua nella direzione della freccia e una semiretta tratteggiata dal lato opposto.
	\end{enumerate}
\end{procedurat}
\begin{figure}
	\centering
	\begin{tikzpicture}
	\draw[ -triangle 90](0,0)--(5,0);
	\draw(2,0)--(2,1);
	%%%%%soluzioni
	%%%sinistra	
	\draw[dashed](2,1)--(5,1);
	%%destra
	\draw(2,1)--(0,1);
	\node at (2,-0.5) {1};
	\end{tikzpicture}
	\captionsetup{format=grafico,list=no}
	\caption[]{Disequazione\nobs\vref{equ:PrimoGradoDisequazione1}}
	\label{fig:esempioDisequazioniPgradografico1}
\end{figure}
\section{Disequazioni di secondo grado}
\section{Disequazioni fratte}
\section{Disequazioni di grado superiore al secondo}
\section{Sistemi di disequazioni intere e fratte}