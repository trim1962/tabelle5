\chapter{Disequazioni}
 \section{Diseguaglianze}
\label{sec:Disequglianze}
Iniziamo con un po' di vocabolario. La tabella~\vref{tab:disuguaglianze} mostra le possibili disuguaglianze\index{Disuguaglianza} e il modo corretto di  leggerle.
\begin{table}
	\centering
	\begin{tabular}{lcll}
		\toprule
		<&$a<b$&minore stretto&<<a è minore di b>>\\
		>&$a>b$&maggiore stretto& <<a è maggiore di b>>\\
		$\leq$&$a\leq b$&minore o uguale& <<a è minore di b>> o <<a è uguale a b>> \\
		$\geq$&$a\geq b$&maggiore o uguale&<<a è maggiore di b>> o <<a è uguale a b>>\\
		\bottomrule
	\end{tabular}
	\caption{Disuguaglianze}
	\label{tab:disuguaglianze}
\end{table}

\section{Disequazioni di primo grado}
\label{sec:Disequuazionidiprimogrado}
\begin{definizionet}{}{}
	Una disequazione\index{Disequazione} è una diseguaglianza\index{Disuguaglianza} in cui compare un'incognita.
\end{definizionet}
\begin{definizionet}{Forma normale}{}
	Una disequazione di primo grado è in forma normale\index{Disequazione!forma normale} se è scritta in una di queste forme
	\begin{equation}
	ax\left\{ \begin{aligned}
	<b\\
	\leq b\\
	\geq b\\
	>b
	\end{aligned}\right .   
	\end{equation}
\end{definizionet}
\section{Disequazioni di secondo grado}
\section{Disequazioni fratte}
\section{Disequazioni di grado superiore al secondo}
\section{Sistemi di disequazioni intere e fratte}