\chapter{Asintoti}
Un asintoto è una retta che, in maniera semplice, a cui si avvicina una curva senza mai toccarla. vi sono tre tipi di asintoti. 
\section{Asintoti verticali}
Gli asintoti verticali, per una funzione razionale fratta, sono legati ai valori del dominio. L'asintoto ha equazione \[ x=k\].
Iniziamo con un esempio per determinare un asintoto verticale\index{Asintoto!verticale} 
consideriamo la funzione 
\begin{esempiot}{Asintoto verticale}{}
\[f(x)=\dfrac{x^2-x-6}{x^2+x-6} \]
\end{esempiot}
Determiniamo il dominio\index{Funzione!dominio}
della funzione. In una razionale fratta il dominio si determina verificando quando il denominatore si annulla quindi
\begin{align*}
x^2+x-6=&0\\
\intertext{risolvo l'equazione ed ottengo}
&\begin{cases}
x_1=-3\\
x_2=2
\end{cases}
\end{align*}
Quindi il dominio della funzione è $\R-\lbrace2,-3\rbrace$ Abbiamo due candidati per gli asintoti verticali.