\chapter{Asintoti}
Un asintoto è una retta che, in maniera semplice, a cui si avvicina una curva senza mai toccarla. Vi sono tre tipi di asintoti. 
\section{Asintoti verticali}
Gli asintoti verticali, per una funzione razionale fratta, sono legati ai valori del dominio.
Una funzione ha un asintoto verticale se 
\[\lim_{x\to k}f(x)=\infty\]
L'asintoto ha equazione \[ x=k\]
Normalmente vi sono quattro tipi di  asintoti che mostreremo tramite esempi.
Iniziamo \index{Asintoto!verticale} 
considerando la funzione 
\begin{esempiot}{Asintoto verticale}{ass1}
\[f(x)=\dfrac{x^2-x-6}{x^2+x-6} \]
\end{esempiot}
Determiniamo il dominio\index{Funzione!dominio}
della funzione. In una razionale fratta il dominio si determina verificando quando il denominatore si annulla quindi
\begin{align*}
x^2+x-6=&0\\
\intertext{risolvo l'equazione ed ottengo}
&\begin{cases}
x_1=-3\\
x_2=2
\end{cases}
\end{align*}
Quindi il dominio della funzione è $\R-\lbrace2,-3\rbrace$ Abbiamo due candidati per gli asintoti verticali.
Calcolo il limite \[ \lim_{x\to 2}\dfrac{x^2-x-6}{x^2+x-6}=\dfrac{-4}{0}=\infty\] Valutiamo i limiti destro e sinistro. 
\begin{align*}
\lim_{x\to 2^+}\dfrac{x^2-x-6}{x^2+x-6}=&\\
\intertext{dato che}
4^+-2^+-6<&0\\
4^++2^+-6>&0\\
\lim_{x\to 2^+}\dfrac{x^2-x-6}{x^2+x-6}=&-\infty\\
\end{align*}
\begin{align*}
\lim_{x\to 2^-}\dfrac{x^2-x-6}{x^2+x-6}=&\\
\intertext{dato che}
4^--2^+-6<&0\\
4^-+2^--6<&0\\
\lim_{x\to 2^-}\dfrac{x^2-x-6}{x^2+x-6}=&+\infty\\
\end{align*}
Quindi \[x=2\] è un asintoto verticale.
L'asintoto è del tipo più infinito meno infinito come il grafico~\vref{fig:asintotopm}
\begin{figure}
	\centering
\includestandalone{grafici/asintotoPM}
\captionsetup{format=grafico}
	\caption[Asintoto piu infinito meno infinito]{Asintoto piu infinito meno infinito}
	\label{fig:asintotopm}
\end{figure}

Calcolo il limite \[\lim_{x\to -3}\dfrac{x^2-x-6}{x^2+x-6}=\dfrac{6}{0}=\infty\] Valutiamo i limiti destro e sinistro. 
\begin{align*}
\lim_{x\to -3^+}\dfrac{x^2-x-6}{x^2+x-6}=&\\
\intertext{dato che}
(-3^+)^2<&9\\
(-3^+)^2-(-3^+)-6>&0\\
(-3^+)^2+(-3^+)-6<&0\\
\lim_{x\to -3^+}\dfrac{x^2-x-6}{x^2+x-6}=&-\infty\\
\end{align*}
\begin{align*}
\lim_{x\to -3^-}\dfrac{x^2-x-6}{x^2+x-6}=&\\
\intertext{dato che}
(-3^-)^2>&9\\
(-3^-)^2-(-3^-)-6>&0\\
(-3^-)^2+(-3^-)-6>&0\\
\lim_{x\to -3^-}\dfrac{x^2-x-6}{x^2+x-6}=&+\infty\\
\end{align*}
Quindi \[x=-3\] è un asintoto verticale.
\begin{esempiot}{Asintoto verticale}{AssMP}
	\[f(x)=\dfrac{2-x}{x-1}\]
\end{esempiot}
Dato che la funzione è una razionale fratta ne calcolo il dominio ponendo il denominatore uguale a zero\[x-1=0\]Ha soluzione $x=1$ quindi il domino è $\R-\lbrace1\rbrace$

Calcoliamo il limite\[\lim_{x\to 1}\dfrac{2-x}{x-1}=\dfrac{2-1}{0}=\infty\]
Valutiamo i limiti destro e sinistro
\begin{align*}
\lim_{x\to 1^+}\dfrac{2-x}{x-1}=&\\
\intertext{dato che}
2-1^+>&0\\
1^+-1>&0\\
\intertext{avremo che}
\lim_{x\to 1^+}\dfrac{2-x}{x-1}=&+\infty\\
\lim_{x\to 1^-}\dfrac{2-x}{x-1}=&\\
\intertext{dato che}
2-1^->&0\\
1^--1<&0\\
\intertext{avremo che}
\lim_{x\to 1^-}\dfrac{2-x}{x-1}=&-\infty\\
\end{align*}
Quindi \[x=2\] è un asintoto verticale.
L'asintoto è del tipo meno infinito più infinito come il grafico~\vref{fig:asintotomp}
\begin{figure}
	\centering
	\includestandalone{grafici/asintotoMP}
	\captionsetup{format=grafico}
	\caption[Asintoto meno infinito più infinito]{Asintoto meno infinito più infinito}
	\label{fig:asintotomp}
\end{figure}
Consideriamo un altra  funzione
\begin{esempiot}{Asintoto verticale}{AsintotoPP}
	\[f(x)=\dfrac{x^2+1}{x^2-2x-1} \]
\end{esempiot}\index{Asintoto!verticale}
La funzione è una razionale fratta per determinarne il dominio pongo il denominatore uguale a zero e risolvo
\[x^2-2x-1=0\]Ha soluzione $x=1$ quindi il domino è $\R-\lbrace1\rbrace$

Calcoliamo il limite\[\lim_{x\to 1}\dfrac{x^2+1}{x^2-2x-1}=\dfrac{2}{0}=\infty\]
Valutiamo i limiti destro e sinistro
\begin{align*}
\lim_{x\to 1^+}\dfrac{x^2+1}{x^2-2x-1}=&\\
\intertext{dato che}
(1^+)^2+1>&0\\
(1^+)^2 -2\cdot 1^+ +1>&0\\
\intertext{quindi}
\lim_{x\to 1^+}\dfrac{x^2+1}{x^2-2x-1}=&+\infty\\
\intertext{Valuto ora}
\lim_{x\to 1^-}\dfrac{x^2+1}{x^2-2x-1}=&\\
\intertext{dato che}
(1^+)^2+1>&0\\
(1^+)^2 -2\cdot 1^+ +1>&0\\
\intertext{avremo che}
\lim_{x\to 1^-}\dfrac{x^2+1}{x^2-2x-1}=&+\infty\\
\end{align*}
Quindi \[x=1\] è un asintoto verticale.
L'asintoto è del tipo più infinito più infinito come il grafico~\vref{fig:asintotopp}
\begin{figure}
	\centering
	\includestandalone{grafici/asintotoPP}
	\captionsetup{format=grafico}
	\caption[Asintoto più infinito più infinito]{Asintoto piu infinito più infinito}
	\label{fig:asintotopp}
\end{figure}

Consideriamo un altra  funzione
\begin{esempiot}{Asintoto verticale}{AsintotoMM}
	\[f(x)=\dfrac{x-1}{x^2} \]
\end{esempiot}\index{Asintoto!verticale}
La funzione è una razionale fratta per determinarne il dominio pongo il denominatore uguale a zero e risolvo
\[x^2=0\]Ha soluzione $x=0$ quindi il domino è $\R-\lbrace0\rbrace$

Calcoliamo il limite\[\lim_{x\to 0}\dfrac{x-1}{x^2}=\dfrac{-1}{0}=\infty\]
Valutiamo i limiti destro e sinistro
\begin{align*}
\lim_{x\to 0^+}\dfrac{x-1}{x^2}=&\\
\intertext{dato che}
(0^+)^2>&0\\
0^+ -1<&0\\
\intertext{quindi}
\lim_{x\to 0^+}\dfrac{x-1}{x^2}=&-\infty\\
\intertext{Valuto ora}
\lim_{x\to 0^-}\dfrac{x-1}{x^2}=&\\
\intertext{dato che}
(0^-)^2>&0\\
0^- -1<&0\\
\intertext{avremo che}
\lim_{x\to 0^-}\dfrac{x-1}{x^2}=&-\infty\\
\end{align*}
Quindi \[x=0\] è un asintoto verticale.
L'asintoto è del tipo meno infinito meno infinito come il grafico~\vref{fig:asintotomm}
\begin{figure}
	\centering
	\includestandalone{grafici/asintotomm}
	\captionsetup{format=grafico}
	\caption[Asintoto meno infinito meno infinito]{Asintoto meno infinito meno infinito}
	\label{fig:asintotomm}
\end{figure}

Non tutte le funzioni razionali fratte hanno asintoti verticali
\begin{cesempiot}{Funzione senza asintoti verticali}{ass2}
	\[f(x)=\dfrac{x^2+3x}{x^2-x+1} \]
\end{cesempiot}
Determino il dominio
\begin{align*}
x^2-x+1=&0\\
\intertext{risolvo l'equazione ed ottengo}
x_{1,2}=\dfrac{1+\pm\sqrt{1-4}}{2}
\end{align*}
L'equazione non ha soluzioni  reali quindi è sempre definita, non ha asintoti verticali.

Negli esempi precedenti gli asintoti verticali, quando esistono, compaiono sempre in coppia tuttavia possiamo avere casi in cui abbiamo un solo asintoto verticale. Consideriamoil seguente esempio
\begin{cesempiot}{Asintoto solo destro}{Asintotosolodestro}
\[ f(x)=\begin{cases}
1-x^3&x\leq 1\\
\dfrac{1}{x-1}&x>1
\end{cases}\]
\end{cesempiot}
\begin{figure}
	\centering
	\includestandalone{grafici/asintotosolodestro}
	\captionsetup{format=grafico}
	\caption[Asintoto solo destro]{Asintoto solo destro}
	\label{fig:asintotosolodestro}
\end{figure}

Questa funzione di cui il grafico~\vref{fig:asintotosolodestro} non è definita per $x=1$ quindi il dominio è $\R-\lbrace0\rbrace$ abbiamo
\begin{align*}
\lim_{x\to 1^-}f(x)=&0\\
\lim_{x\to 1^+}f(x)=&+\infty\\
\end{align*}
Quindi dato che il limite sinistro è finito mentre il destro è infinito, la funzione ha un solo un limite destro. 
\section{Asintoti orizzontali}
Una funzione $f(x)$ ha un asintoto orizzontale se è verificato che \[\lim_{x\to \infty}f(x)=l \]
Iniziamo con un esempio di funzione che ha un asintoto orizzontale
\begin{esempiot}{Asintoti orizzontali}{ass3}
	\[f(x)=\dfrac{3x^4+2x^3+1}{2x^4+2x+1}\]
\end{esempiot}\index{Asintoto!orizzontale}
Calcolo il limite
\begin{align*}\index{Forma!indeterminata!infinito su infinito}
\lim_{x\to \infty}\dfrac{3x^4+2x^3+1}{2x^4+2x+1}=&\dfrac{\infty}{\infty}\\
\intertext{ottengo una forma indeterminata infinto su infinito quindi, raccogliendo i termini di grado maggiore a numeratore e denominatore}
=&\lim_{x\to \infty}\dfrac{x^4\left(3+\dfrac{2x^2}{x^4}+\dfrac{1}{x^4}\right)}{x^4\left(2+\dfrac{2x}{x^4}+\dfrac{1}{x^4}\right)}\\
\intertext{semplificando}
=&\lim_{x\to \infty}\dfrac{x^4\left(3+\dfrac{2x}{x^2}+\dfrac{1}{x^4}\right)}{x^4\left(2+\dfrac{2}{x^3}+\dfrac{1}{x^4}\right)}\\
\intertext{dato che}
\lim_{x\to \infty}\dfrac{2}{x^2}=&0\\
\lim_{x\to \infty}\dfrac{1}{x^4}=&0\\
\lim_{x\to \infty}\dfrac{2}{x^3}=&0\\
\dfrac{x^4}{x^4}=&1\\
\intertext{ottengo}
=&\lim_{x\to \infty}\dfrac{3}{2}=\dfrac{3}{2}\\
\end{align*}
Abbiamo un limite, e di conseguenza un asintoto orizzontale\[y=\dfrac{3}{2}\]  che si comporta come nel grafico~\vref{fig:asintotorizzo}
\begin{figure}
	\centering
	\includestandalone{grafici/asintotorizzo}
	\captionsetup{format=grafico}
	\caption[Asintoto orizzontale]{Asintoto orizzontale}
	\label{fig:asintotorizzo}
\end{figure}
\begin{esempiot}{Asintoti orizzontali}{ass4}
	\[f(x)=\dfrac{x^2+x+1}{2x^3+2x+3}\]
\end{esempiot}
Per determinare l'asintoto orizzontale calcolo il limite 
\begin{align*}\index{Forma!indeterminata!infinito su infinito}
\lim_{x\to \infty}\dfrac{x^2+x+1}{2x^3+2x+3}=&\dfrac{\infty}{\infty}\\
\intertext{ottengo una forma indeterminata infinto su infinito quindi, raccogliendo i termini di grado maggiore a numeratore e denominatore}
=&\lim_{x\to \infty}\dfrac{x^2\left(1+\dfrac{x}{x^2}+\dfrac{1}{x^2}\right)}{x^3\left(2+\dfrac{2x}{x^3}+\dfrac{3}{x^3}\right)}\\
\intertext{semplificando}
=&\lim_{x\to \infty}\dfrac{x^2\left(1+\dfrac{1}{x}+\dfrac{1}{x^2}\right)}{x^3\left(2+\dfrac{2}{x^2}+\dfrac{3}{x^3}\right)}\\
\intertext{dato che}
\lim_{x\to \infty}\dfrac{1}{x}=&0\\
\lim_{x\to \infty}\dfrac{1}{x^2}=&0\\
\lim_{x\to \infty}\dfrac{2}{x^2}=&0\\
\lim_{x\to \infty}\dfrac{3}{x^3}=&0\\
\dfrac{x^2}{x^3}=&\dfrac{1}{x}\\
\intertext{ottengo}
=&\lim_{x\to \infty}\dfrac{1}{x}\cdot\dfrac{1}{2}=0\\
\end{align*}
Il limite esiste ed è finito quindi la funzione ha un asintoto orizzontale che è \[y=0\] come mostrato nel grafico~\vref{fig:asintotorizzodue}
\begin{figure}
	\centering
	\includestandalone{grafici/asintotorizzodue}
	\captionsetup{format=grafico}
	\caption[Asintoto orizzontale]{Asintoto orizzontale}
	\label{fig:asintotorizzodue}
\end{figure}
\begin{cesempiot}{Funzione senza asintoto orizzontale}{ass5}
	\[f(x)=\dfrac{2x^5+x^2+1}{3x^2+1}\]
\end{cesempiot}
Per determinare l'asintoto orizzontale calcolo il limite 
\begin{align*}\index{Forma!indeterminata!infinito su infinito}
\lim_{x\to \infty}\dfrac{2x^5+x^2+1}{3x^2+1}=&\dfrac{\infty}{\infty}\\
\intertext{ottengo una forma indeterminata infinto su infinito quindi, raccogliendo i termini di grado maggiore a numeratore e denominatore}
=&\lim_{x\to \infty}\dfrac{x^5\left(2+\dfrac{x^2}{x^5}+\dfrac{1}{x^5}\right)}{x^2\left(3+\dfrac{1}{x^2}\right)}\\
\intertext{semplificando}
=&\lim_{x\to \infty}\dfrac{x^5\left(2+\dfrac{1}{x^3}+\dfrac{1}{x^5}\right)}{x^2\left(3+\dfrac{1}{x^2}\right)}\\
\intertext{dato che}
\lim_{x\to \infty}\dfrac{1}{x^3}=&0\\
\lim_{x\to \infty}\dfrac{1}{x^5}=&0\\
\lim_{x\to \infty}\dfrac{1}{x^2}=&0\\
\dfrac{x^5}{x^2}=&x^3\\
\intertext{ottengo}
=&\lim_{x\to \infty}x^3\cdot\dfrac{2}{3}=\infty\\
\end{align*}
Il limite non esiste  finito quindi la funzione non ha un asintoto orizzontale. Il suo comportamento è mostrato dal grafico~\vref{fig:asintotorizzotre}.
\begin{figure}
	\centering
	\includestandalone{grafici/asintotorizzotre}
	\captionsetup{format=grafico}
	\caption[Asintoto orizzontale]{Asintoto orizzontale}
	\label{fig:asintotorizzotre}
\end{figure}

I tre precedenti esempi ci permettono di definire una regola per determinare se una funzione abbia o non abbia un asintoto orizzontale. Una funziona razionale fratta ha un asintoto orizzontale se il grado del denominatore è maggiore o uguale al grado del numeratore. 
\section{Asintoti obliqui}
Una funzione $f(x)$ ha un asintoto obliquo se esiste una retta $y=mx+q$ che verifica \[\lim_{x\to \infty}[f(x)-(mx+q)]=0 \]
all'atto pratico se esistono e sono finiti entrambi i seguenti limiti
\begin{align*}
m=&\lim_{x\to \infty}\dfrac{f(x)}{x}\\
q=&\lim_{x\to \infty}[f(x)-mx]
\end{align*}\index{Asintoto!obliquo}
\begin{esempiot}{Asintoti obliqui}{ass6}
	\[f(x)=\dfrac{2x^2+x+1}{x+1}\]
\end{esempiot}
\begin{figure}
	\centering
	\includestandalone[scale=0.3]{grafici/asintotoobliquouno}
	\captionsetup{format=grafico}
	\caption[Asintoto obliquo]{Asintoto obliquo}
	\label{fig:asintoobliquouno}
\end{figure}
Per determinare l'asintoto obliquo iniziamo a calcolare il limite
\begin{align*}
m=&\lim_{x\to \infty}\dfrac{f(x)}{x}\\
=&\lim_{x\to \infty}\dfrac{2x^2+x+1}{x+1}\cdot\dfrac{1}{x}\\
=&\lim_{x\to \infty}\dfrac{2x^2+x+1}{x^2+x}\\
=&\lim_{x\to \infty}\dfrac{x^2\left(2+\dfrac{x}{x^2}+\dfrac{1}{x^2}\right)}{x^2\left(1+\dfrac{x}{x^2}\right)}\\
\intertext{semplifico}
=&\lim_{x\to \infty}\dfrac{x^2\left(2+\dfrac{1}{x}+\dfrac{1}{x^2}\right)}{x^2\left(1+\dfrac{1}{x}\right)}\\
\intertext{dato che}
\dfrac{x^2}{x^2}=&1\\
\lim_{x\to \infty}\dfrac{1}{x^2}=&0\\
\lim_{x\to \infty}\dfrac{1}{x}=&0\\
=&\lim_{x\to \infty}1\cdot 2=2
\end{align*} 
Quindi il coefficiente angolare $m$ è due
\[m=2\]
Resta da determinare il coefficiente $q$
\begin{align*}
q=&\lim_{x\to \infty}[f(x)-mx]\\
=&\lim_{x\to \infty}[\dfrac{2x^2+x+1}{x+1}-2x]
\intertext{sommo}
=&\lim_{x\to \infty}[\dfrac{2x^2+x+1-2x^2-2x}{x+1}]\\
\intertext{semplificando}
=&\lim_{x\to \infty}[\dfrac{-x+1}{x+1}]\\
=&\lim_{x\to \infty}[\dfrac{x\left(-1+\dfrac{1}{x}\right)}{x\left(1+\dfrac{1}{x}\right)}]\\
\intertext{dato che}
\dfrac{x}{x}=&1\\
\lim_{x\to \infty}\dfrac{1}{x}=&0\\
=&\dfrac{x^2}{x^2}=&1\\
=&\lim_{x\to \infty}1\cdot\dfrac{-1}{1}=-1\\
\end{align*}
Quindi l'asintoto obliquo è
\[y=2x-1\]
Il grafico~\vref{fig:asintoobliquouno} mostra l'andamento.